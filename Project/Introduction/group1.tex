\documentclass{article}
\usepackage[utf8]{inputenc}
\usepackage{setspace}
\usepackage{tikz}
\usetikzlibrary{positioning}
\usepackage{amsfonts}
\usepackage{amssymb}
\usepackage{amsmath}
\usepackage{amsthm}
\usepackage{systeme}
\usepackage{mathtools}
\usepackage{hyperref}
\usepackage{venndiagram}
\usepackage{pgfplots}
\usetikzlibrary{pgfplots.statistics}
\pgfplotsset{compat=newest}
\usepackage{apacite}
\bibliographystyle{apacite}

\author{Zian Kang, Jeffery Shu}

\title{Introduction Report}

\date{Oct. 2024}

\begin{document}

\maketitle

\newpage

\section{Project Introduction}

~

On the roads, we see a wide variety of vehicle types, such as SUVs, hatchbacks, and passenger cars. Every day, countless dealerships sell thousands of different models. As automotive enthusiasts, we have a deep passion for cars. But when we buy cars, we find out that the distribution of car supllies are very unreasonbale: the popular cars always are short in supply, while the unpopular ones have a lot in stock. When connecting our interest with this course, we thought of a topic that might be practically beneficial: If we could understand people's preferences for different car models, could we optimize the distribution of cars in each region, thereby reducing inventory and transportation costs?

~

\section{Topic and Question}

~

And by the doubt of the cars' popularity, we have stated our research topic: Affect of income and region on people's choices of choosing cars. More specifically, our research question is that how can differences in income and region and other aspects overall affect people's choices on cars? To provide a hypothesis for this question, we uses our knowledge about cars and provide a hypothesis for this topic: The more northern people lives and the more money people have and for middle-aged, the more likely they are going to buy SUVs.

~

\section{Data}

~

\url{https://www.kaggle.com/datasets/missionjee/car-sales-report}

~

This is our data source, which have over 20 thousand data and ranging from the first day of 2022 to the last day of 2023, a two year period. The data also covers comsumers aging from 20 to 60, varying differently in incomes.

~

\section{Similar Studies and Relevancy}

~

There have been similar studies conducted in the UK\cite{PIERCE2023363}, but due to differences in countries, habits, and preferences, we are eager to explore how these factors play out in the U.S. To the best of our knowledge, no such study has been done in the U.S., and we aim to apply the statistical knowledge we've gained to present this research.

~

When someone decides to purchase a vehicle, there are numerous influencing factors, such as age, location, purpose, and income.  Our research aims to identify how more quantifiable factors, income and region, affect people's choices of vehicle types.  We believe this is a highly practical study, as it can truly benefit the automotive industry by informing better decision-making.


\newpage

\bibliography{introduction.bib}

\end{document}